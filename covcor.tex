% Options for packages loaded elsewhere
\PassOptionsToPackage{unicode}{hyperref}
\PassOptionsToPackage{hyphens}{url}
\PassOptionsToPackage{dvipsnames,svgnames,x11names}{xcolor}
%
\documentclass[
  12pt,
  letterpaper,
  DIV=11,
  numbers=noendperiod]{scrartcl}

\usepackage{amsmath,amssymb}
\usepackage{iftex}
\ifPDFTeX
  \usepackage[T1]{fontenc}
  \usepackage[utf8]{inputenc}
  \usepackage{textcomp} % provide euro and other symbols
\else % if luatex or xetex
  \usepackage{unicode-math}
  \defaultfontfeatures{Scale=MatchLowercase}
  \defaultfontfeatures[\rmfamily]{Ligatures=TeX,Scale=1}
\fi
\usepackage{lmodern}
\ifPDFTeX\else  
    % xetex/luatex font selection
    \setmainfont[]{Times New Roman}
\fi
% Use upquote if available, for straight quotes in verbatim environments
\IfFileExists{upquote.sty}{\usepackage{upquote}}{}
\IfFileExists{microtype.sty}{% use microtype if available
  \usepackage[]{microtype}
  \UseMicrotypeSet[protrusion]{basicmath} % disable protrusion for tt fonts
}{}
\makeatletter
\@ifundefined{KOMAClassName}{% if non-KOMA class
  \IfFileExists{parskip.sty}{%
    \usepackage{parskip}
  }{% else
    \setlength{\parindent}{0pt}
    \setlength{\parskip}{6pt plus 2pt minus 1pt}}
}{% if KOMA class
  \KOMAoptions{parskip=half}}
\makeatother
\usepackage{xcolor}
\setlength{\emergencystretch}{3em} % prevent overfull lines
\setcounter{secnumdepth}{5}
% Make \paragraph and \subparagraph free-standing
\makeatletter
\ifx\paragraph\undefined\else
  \let\oldparagraph\paragraph
  \renewcommand{\paragraph}{
    \@ifstar
      \xxxParagraphStar
      \xxxParagraphNoStar
  }
  \newcommand{\xxxParagraphStar}[1]{\oldparagraph*{#1}\mbox{}}
  \newcommand{\xxxParagraphNoStar}[1]{\oldparagraph{#1}\mbox{}}
\fi
\ifx\subparagraph\undefined\else
  \let\oldsubparagraph\subparagraph
  \renewcommand{\subparagraph}{
    \@ifstar
      \xxxSubParagraphStar
      \xxxSubParagraphNoStar
  }
  \newcommand{\xxxSubParagraphStar}[1]{\oldsubparagraph*{#1}\mbox{}}
  \newcommand{\xxxSubParagraphNoStar}[1]{\oldsubparagraph{#1}\mbox{}}
\fi
\makeatother


\providecommand{\tightlist}{%
  \setlength{\itemsep}{0pt}\setlength{\parskip}{0pt}}\usepackage{longtable,booktabs,array}
\usepackage{calc} % for calculating minipage widths
% Correct order of tables after \paragraph or \subparagraph
\usepackage{etoolbox}
\makeatletter
\patchcmd\longtable{\par}{\if@noskipsec\mbox{}\fi\par}{}{}
\makeatother
% Allow footnotes in longtable head/foot
\IfFileExists{footnotehyper.sty}{\usepackage{footnotehyper}}{\usepackage{footnote}}
\makesavenoteenv{longtable}
\usepackage{graphicx}
\makeatletter
\newsavebox\pandoc@box
\newcommand*\pandocbounded[1]{% scales image to fit in text height/width
  \sbox\pandoc@box{#1}%
  \Gscale@div\@tempa{\textheight}{\dimexpr\ht\pandoc@box+\dp\pandoc@box\relax}%
  \Gscale@div\@tempb{\linewidth}{\wd\pandoc@box}%
  \ifdim\@tempb\p@<\@tempa\p@\let\@tempa\@tempb\fi% select the smaller of both
  \ifdim\@tempa\p@<\p@\scalebox{\@tempa}{\usebox\pandoc@box}%
  \else\usebox{\pandoc@box}%
  \fi%
}
% Set default figure placement to htbp
\def\fps@figure{htbp}
\makeatother
% definitions for citeproc citations
\NewDocumentCommand\citeproctext{}{}
\NewDocumentCommand\citeproc{mm}{%
  \begingroup\def\citeproctext{#2}\cite{#1}\endgroup}
\makeatletter
 % allow citations to break across lines
 \let\@cite@ofmt\@firstofone
 % avoid brackets around text for \cite:
 \def\@biblabel#1{}
 \def\@cite#1#2{{#1\if@tempswa , #2\fi}}
\makeatother
\newlength{\cslhangindent}
\setlength{\cslhangindent}{1.5em}
\newlength{\csllabelwidth}
\setlength{\csllabelwidth}{3em}
\newenvironment{CSLReferences}[2] % #1 hanging-indent, #2 entry-spacing
 {\begin{list}{}{%
  \setlength{\itemindent}{0pt}
  \setlength{\leftmargin}{0pt}
  \setlength{\parsep}{0pt}
  % turn on hanging indent if param 1 is 1
  \ifodd #1
   \setlength{\leftmargin}{\cslhangindent}
   \setlength{\itemindent}{-1\cslhangindent}
  \fi
  % set entry spacing
  \setlength{\itemsep}{#2\baselineskip}}}
 {\end{list}}
\usepackage{calc}
\newcommand{\CSLBlock}[1]{\hfill\break\parbox[t]{\linewidth}{\strut\ignorespaces#1\strut}}
\newcommand{\CSLLeftMargin}[1]{\parbox[t]{\csllabelwidth}{\strut#1\strut}}
\newcommand{\CSLRightInline}[1]{\parbox[t]{\linewidth - \csllabelwidth}{\strut#1\strut}}
\newcommand{\CSLIndent}[1]{\hspace{\cslhangindent}#1}

\usepackage{tcolorbox}
\usepackage{amssymb}
\usepackage{yfonts}
\usepackage{bm}


\newtcolorbox{greybox}{
  colback=white,
  colframe=blue,
  coltext=black,
  boxsep=5pt,
  arc=4pt}
  
\newcommand{\sectionbreak}{\clearpage}

 
\newcommand{\ds}[4]{\sum_{{#1}=1}^{#3}\sum_{{#2}=1}^{#4}}
\newcommand{\us}[3]{\mathop{\sum\sum}_{1\leq{#2}<{#1}\leq{#3}}}

\newcommand{\ol}[1]{\overline{#1}}
\newcommand{\ul}[1]{\underline{#1}}

\newcommand{\amin}[1]{\mathop{\text{argmin}}_{#1}}
\newcommand{\amax}[1]{\mathop{\text{argmax}}_{#1}}

\newcommand{\ci}{\perp\!\!\!\perp}

\newcommand{\mc}[1]{\mathcal{#1}}
\newcommand{\mb}[1]{\mathbb{#1}}
\newcommand{\mf}[1]{\mathfrak{#1}}

\newcommand{\eps}{\epsilon}
\newcommand{\lbd}{\lambda}
\newcommand{\alp}{\alpha}
\newcommand{\df}{=:}
\newcommand{\am}[1]{\mathop{\text{argmin}}_{#1}}
\newcommand{\ls}[2]{\mathop{\sum\sum}_{#1}^{#2}}
\newcommand{\ijs}{\mathop{\sum\sum}_{1\leq i<j\leq n}}
\newcommand{\jis}{\mathop{\sum\sum}_{1\leq j<i\leq n}}
\newcommand{\sij}{\sum_{i=1}^n\sum_{j=1}^n}
	
\KOMAoption{captions}{tableheading}
\makeatletter
\@ifpackageloaded{caption}{}{\usepackage{caption}}
\AtBeginDocument{%
\ifdefined\contentsname
  \renewcommand*\contentsname{Table of contents}
\else
  \newcommand\contentsname{Table of contents}
\fi
\ifdefined\listfigurename
  \renewcommand*\listfigurename{List of Figures}
\else
  \newcommand\listfigurename{List of Figures}
\fi
\ifdefined\listtablename
  \renewcommand*\listtablename{List of Tables}
\else
  \newcommand\listtablename{List of Tables}
\fi
\ifdefined\figurename
  \renewcommand*\figurename{Figure}
\else
  \newcommand\figurename{Figure}
\fi
\ifdefined\tablename
  \renewcommand*\tablename{Table}
\else
  \newcommand\tablename{Table}
\fi
}
\@ifpackageloaded{float}{}{\usepackage{float}}
\floatstyle{ruled}
\@ifundefined{c@chapter}{\newfloat{codelisting}{h}{lop}}{\newfloat{codelisting}{h}{lop}[chapter]}
\floatname{codelisting}{Listing}
\newcommand*\listoflistings{\listof{codelisting}{List of Listings}}
\makeatother
\makeatletter
\makeatother
\makeatletter
\@ifpackageloaded{caption}{}{\usepackage{caption}}
\@ifpackageloaded{subcaption}{}{\usepackage{subcaption}}
\makeatother

\usepackage{bookmark}

\IfFileExists{xurl.sty}{\usepackage{xurl}}{} % add URL line breaks if available
\urlstyle{same} % disable monospaced font for URLs
\hypersetup{
  pdftitle={The Covariance of Covariances},
  pdfauthor={Jan de Leeuw},
  colorlinks=true,
  linkcolor={blue},
  filecolor={Maroon},
  citecolor={Blue},
  urlcolor={Blue},
  pdfcreator={LaTeX via pandoc}}


\title{The Covariance of Covariances}
\author{Jan de Leeuw}
\date{February 23, 2025}

\begin{document}
\maketitle
\begin{abstract}
We derive the classical formula for the first and second order moments
of covariance using discrete variable calculations. In addition the
equally classical first order asymptotics for covariances and
correlations is discussed. And finally we derive an exact expression for
the covariances of covariances.
\end{abstract}

\renewcommand*\contentsname{Table of contents}
{
\hypersetup{linkcolor=}
\setcounter{tocdepth}{3}
\tableofcontents
}

\sectionbreak

\section*{Note}\label{note}
\addcontentsline{toc}{section}{Note}

This is a working manuscript which will be expanded/updated frequently.
All suggestions for improvement are welcome. All Rmd, tex, html, pdf, R,
and C files are in the public domain. Attribution will be appreciated,
but is not required. The files can be found at
\url{https://github.com/deleeuw/covcor}

\section*{Notation}\label{notation}
\addcontentsline{toc}{section}{Notation}

\begin{itemize}
\tightlist
\item
  We use the ``Dutch Convention'' of underlining random variables
  (Hemelrijk (\citeproc{ref-hemelrijk_66}{1966})).
\item
  The symbol \(:=\) is used for definitions.
\item
  A sequence \(x_n\) is \(o(n^{-r})\) if \(n^rx_n\) converges to zero.
\item
  A sequence \(\ul{x}_n\) of random variables is \(o_p(n^{-r})\) if
  \(n^r\ul{x}_n\) converges to zero in probability (Mann and Wald
  (\citeproc{ref-mann_wald_43}{1943})).
\end{itemize}

\sectionbreak

\section{Data}\label{data}

In this paper the data are \(n\) observations on \(m\) discrete
numerical variables. Variable \(j\) has \(k_j\) \emph{categories} (or
\emph{levels}). Thus there are \(K:=\smash{\prod_{j=1}^m}k_j\) possible
\emph{profiles}.

Suppose we have \(p\) real vectors \(x_1,\cdots,x_p\) of length \(m\),
which we call \emph{profiles}. Profiles are the possible outcomes of a
measurement of \(m\) variables, and in our framework the number of
profiles is finite.

Define a random vector \(\ul{e}\), which takes as its values the \(p\)
unit vectors \(e_1,\cdots,e_p\). Unit vector \(e_r\) has all its
elements equal to zero, except for element \(r\), which is equal to one.
Define \(\pi_r:=\text{prob}(\ul{e}=e_r)\) and the random vector \[
\ul{x}=\sum_{r=1}^p\ul{e}_rx_r
\] which takes the value \(x_r\) with probability \(\pi_r\).

Next define the raw product moments, or product moments about zero, of
order \(t\) as \[
\mu_{i_1\cdots i_t}:=\sum_{r=1}^p \pi_r\prod_{s=1}^tx_{ri_s},
\] and the centered product moments, or product moments around the mean,
by

\[
\sigma_{i_1\cdots i_t}:=\sum_{r=1}^p \pi_r\prod_{s=1}^t(x_{ri_s}-\mu_{i_s}).
\] Note that some of the subscripts \(i_1,\cdots,i_t\) can be equal

Next, suppose we have \(n\) independent copies
\(\ul{e}_1,\cdots\ul{e}_n\) of \(\ul{e}\) . Suppose \[
\ul{n}=\sum_{\nu=1}^n\ul{e}_\nu\\
\ul{p}=\frac{1}{n}\ul{n}
\] are the vectors of, respectively, \emph{frequencies} and
\emph{proportions} of the profiles. the covariance of variables \(i\)
and \(j\) as \[
\ul{c}_{ij}:=\sum_\nu \ul{p}_\nu x_{\nu i}x_{\nu j}-\sum_\nu \ul{p}_\nu x_{\nu i}\sum_\eta\ul{p}_\eta x_{\eta j}
\] We want to compute the expected value of the covariances
\(\ul{c}_{ij}\) and the covariance of pairs of covariances
\(\ul{c}_{ij}\) and \(\ul{c}_{kl}\).

\section{Calculations}\label{calculations}

Let \(\ul{\epsilon}_\nu:=\ul{p}_\nu-\pi_\nu\) so that \[
\mathbf{E}(\ul{\epsilon}_\nu)=0,
\] and \[
\mathbf{E}(\ul{\epsilon}_\nu\ul{\epsilon}_\eta)=n^{-1}(\delta^{\nu\eta}\pi_\nu-\pi_\nu\pi_\eta).
\]

Now we can write \[
\ul{c}_{ij}=\sigma_{ij}+\sum_\nu \ul{\epsilon}_\nu(x_{\nu i}-\mu_i)(x_{\nu j}-\mu_j)
-\sum_\nu\sum_\eta\ul{\epsilon}_\nu\ul{\epsilon}_\eta(x_{\nu i}-\mu_i)(x_{\eta j}-\mu_j),
\] with \[
\sigma_{ij}=\sum_\nu\pi_\nu(x_{\nu i}-\mu_i)(x_{\nu j}-\mu_j).
\] and \[
\mu_i=\sum_\nu \pi_\nu x_{\nu i}
\]

It follows that \[
\mathbf{E}(\ul{c}_{ij})=\sigma_{ij}-\frac{1}{n}\sum_\nu\sum_\eta x_{\nu i}x_{\eta j}(\delta^{\nu\eta}\pi_\nu-\pi_\nu\pi_\eta)=\frac{n-1}{n}\sigma_{ij}.
\] Next, define \[
\ul{\delta}_{ij}:=\ul{c}_{ij}-\frac{n-1}{n}\sigma_{ij}.
\] The \(\ul{\delta}_{ij}\) have expectation zero, and \[
\text{COV}(\ul{c}_{ij},\ul{c}_{kl})=\mathbf{E}(\ul{\delta}_{ij}\ul{\delta}_{kl}).
\] We have \[
\ul{\delta}_{ij}=\frac{1}{n}\sigma_{ij}+\sum_\alpha\ul{\epsilon}_\alpha(x_{\alpha i}-\mu_i)(x_{\alpha j}-\mu_j)
-\sum_\gamma\sum_\xi\ul{\epsilon}_\gamma\ul{\epsilon}_\xi(x_{\gamma i}-\mu_i)(x_{\xi j}-\mu_j),
\] \[
\ul{\delta}_{kl}=\frac{1}{n}\sigma_{kl}+\sum_\beta\ul{\epsilon}_\beta(x_{\beta k}-\mu_k)(x_{\beta l}-\mu_l)
-\sum_\nu\sum_\eta\ul{\epsilon}_\nu\ul{\epsilon}_\eta(x_{\nu k}-\mu_k)(x_{\eta l}-\mu_l).
\]

If we multiply \eqref{eq-expd1} and \eqref{eq-expd2} we have nine terms.
Taking expectations of each of these nine terms gives \[
\text{term I}:\frac{1}{n^2}\sigma_{ij}\sigma_{kl}
\] \[
\text{term II}:\frac{1}{n}\sigma_{ij}\sum_\beta\mathbf{E}(\ul{\epsilon}_\beta)(x_{\beta k}-\mu_k)(x_{\beta l}-\mu_l)=0
\] \[
\text{term III}:-\frac{1}{n}\sigma_{ij}\sum_\nu\sum_\eta\mathbf{E}(\ul{\epsilon}_\nu\ul{\epsilon}_\eta) (x_{\nu k}-\mu_k)(x_{\eta l}-\mu_l)=-\frac{1}{n^2}\sigma_{ij}\sigma_{kl}
\] \[
\text{term IV}:\frac{1}{n}\sigma_{kl}\sum_\alpha\mathbf{E}(\ul{\epsilon}_\alpha)(x_{\alpha i}-\mu_i)(x_{\alpha j}-\mu_j)=0
\] \[
\text{term V}:\sum_\alpha\sum_\beta\mathbf{E}(\ul{\epsilon}_\alpha\ul{\epsilon}_\beta)
(x_{\alpha i}-\mu_i)(x_{\alpha j}-\mu_j)(x_{\beta k}-\mu_k)(x_{\beta l}-\mu_l)=\frac{1}{n}(\sigma_{ijkl}-\sigma_{ij}\sigma_{kl})\]

\[
\text{term VI}:-\sum_\alpha\sum_\nu\sum_\eta\mathbf{E}(\ul{\epsilon}_\alpha\ul{\epsilon}_\nu\ul{\epsilon}_\eta)(x_{\alpha i}-\mu_i)(x_{\alpha j}-\mu_j)(x_{\nu k}-\mu_k)(x_{\eta l}-\mu_l)
\]

\[
\text{term VII}:-\frac{1}{n}\sigma_{kl}\sum_\gamma\sum_\xi\mathbf{E}(\ul{\epsilon}_\gamma\ul{\epsilon}_\xi) (x_{\gamma i}-\mu_i)(x_{\xi j}-\mu_j)=-\frac{1}{n^2}\sigma_{ij}\sigma_{kl}
\] \[
\text{term VIII}:-\sum_\beta\sum_\gamma\sum_\xi\mathbf{E}(\ul{\epsilon}_\beta\ul{\epsilon}_\gamma\ul{\epsilon}_\xi)(x_{\beta k}-\mu_k)(x_{\beta l}-\mu_l)(x_{\gamma i}-\mu_i)(x_{\xi j}-\mu_j)
\]

\[
\text{term IX}:\sum_\nu\sum_\eta\sum_\gamma\sum_\xi\mathbf{E}(\ul{\epsilon}_\nu\ul{\epsilon}_\eta\ul{\epsilon}_\gamma\ul{\epsilon}_\xi)(x_{\nu i}-\mu_i)(x_{\eta j}-\mu_j)(x_{\gamma k}-\mu_k)(x_{\xi l}-\mu_l)
\]

From Ouimet (\citeproc{ref-ouimet_21}{2021})
\[\mathbf{E}(\ul{\epsilon}_\alpha\ul{\epsilon}_\nu\ul{\epsilon}_\eta)=\frac{1}{n^2}
\{2\pi_\alpha\pi_\nu\pi_\eta
-\delta^{\alpha\nu}\pi_\alpha\pi_\eta
-\delta^{\nu\eta}\pi_\alpha\pi_\nu
-\delta^{\alpha\eta}\pi_\nu\pi_\eta
+\delta^{\alpha\nu\eta}\pi_\alpha\}.
\] We can now evaluate term VI and term VIII. \[
2\sum_\alpha\sum_\nu\sum_\eta\pi_\alpha\pi_\nu\pi_\eta
(x_{\alpha i}-\mu_i)(x_{\alpha j}-\mu_j)(x_{\nu k}-\mu_k)(x_{\eta l}-\mu_l)=0
\] \[
-\sum_\alpha\sum_\eta\pi_\alpha\pi_\eta(x_{\alpha i}-\mu_i)(x_{\alpha j}-\mu_j)(x_{\alpha k}-\mu_k)(x_{\eta l}-\mu_l)=0
\] \[
-\sum_\alpha\sum_\nu
\pi_\alpha\pi_\nu
(x_{\alpha i}-\mu_i)(x_{\alpha j}-\mu_j)(x_{\nu k}-\mu_k)(x_{\nu l}-\mu_l)=-\sigma_{ij}\sigma_{kl}
\] \[
-\sum_\alpha\sum_\nu
\pi_\alpha\pi_\nu(x_{\alpha i}-\mu_i)(x_{\alpha j}-\mu_j)(x_{\nu k}-\mu_k)(x_{\nu l}-\mu_l)=-\sigma_{ij}\sigma_{kl}
\] \[
+\sum_\alpha
\pi_\alpha(x_{\alpha i}-\mu_i)(x_{\alpha j}-\mu_j)(x_{\alpha k}-\mu_k)(x_{\alpha l}-\mu_l)=\sigma_{ijkl}
\]

\section{Asymptotics}\label{asymptotics}

From

\[
\mathbf{E}(\ul{\delta}_{ij}\ul{\delta}_{kl})=\frac{1}{n}\sum_\nu\sum_\eta(\delta^{\nu\eta}\pi_\nu-\pi_\nu\pi_\eta)
\{x_{\nu i}x_{\nu j}
-\mu_jx_{\nu i}-\mu_ix_{\nu j}\}\{x_{\eta k}x_{\eta l}
-\mu_lx_{\eta k}-\mu_kx_{\eta l}\}=\frac{1}{n}\mu_{ijkl}-
\] \[
x_{\nu i}x_{\nu j}=(x_{\nu i}-\mu_i)(x_{\nu j}-\mu_j)+x_{\nu i}\mu_j+x_{\nu j}\mu_i-\mu_i\mu_j
\] \[x_{\nu i}x_{\nu j}
-\mu_jx_{\nu i}-\mu_ix_{\nu j}=(x_{\nu i}-\mu_i)(x_{\nu j}-\mu_j)-\mu_i\mu_j
\] \begin{multline}
\mathbf{E}(\ul{\delta}_{ij}\ul{\delta}_{kl})=\frac{1}{n}\sum_\nu\sum_\eta(\delta^{\nu\eta}\pi_\nu-\pi_\nu\pi_\eta)
\{(x_{\nu i}-\mu_i)(x_{\nu j}-\mu_j)-\mu_i\mu_j\}\{(x_{\eta k}-\mu_k)(x_{\eta l}-\mu_l)-\mu_k\mu_l\}=\\
\sigma_{ijkl}-\mu_k\mu_l\sigma_{ij}-\mu_i\mu_j\sigma_{kl}+\mu_i\mu_j\mu_k\mu_l-(\sigma_{ij}-\mu_i\mu_j)(\sigma_{kl}-\mu_k\mu_l)=\sigma_{ijkl}-\sigma_{ij}\sigma_{kl}
\end{multline}

If we define \[
\ul{z}_{ij}:=n^\frac12(\ul{s}_{ij}-\sigma_{ij})
\] we have \[
\ul{r}_{ij}=(\sigma_{ij}+n^{-\frac12}z_{ij})(\sigma_{ii}+n^{-\frac12}z_{ii})^{-\frac12}(\sigma_{jj}+n^{-\frac12}z_{jj})^{-\frac12},
\]

which implies \[
\ul{r}_{ij}=\rho_{ij}+n^{-\frac12}\rho_{ij}\left\{\frac{\ul{z}_{ij}}{\sigma_{ij}}-\frac12\frac{\ul{z}_{ii}}{\sigma_{ii}}-\frac12\frac{\ul{z}_{jj}}{\sigma_{jj}}\right\}+o_p(n^{-\frac12}).
\] Multiplying \ldots{} for \(\ul{r}_{ij}\) and \(\ul{r}_{kl}\) and
simplifying gives \begin{multline}
n\text{COV}(\ul{r}_{ij},\ul{r}_{kl})=
\rho_{ij}\rho_{kl}\left\{\frac{\sigma_{ijkl}}{\sigma_{ij}\sigma_{kl}}
-\frac12\left(\frac{\sigma_{ijkk}}{\sigma_{ij}\sigma_{kk}}
+\frac{\sigma_{ijll}}{\sigma_{ij}\sigma_{ll}}
+\frac{\sigma_{iikl}}{\sigma_{ii}\sigma_{kl}}
+\frac{\sigma_{jjkl}}{\sigma_{jj}\sigma_{kl}}\right)\right.\\
\left.+\frac14\left(\frac{\sigma_{iikk}}{\sigma_{ii}\sigma_{kk}}
+\frac{\sigma_{iill}}{\sigma_{ii}\sigma_{ll}}+
+\frac{\sigma_{jjkk}}{\sigma_{jj}\sigma_{kk}}+
+\frac{\sigma_{jjll}}{\sigma_{jj}\sigma_{ll}}\right)\right\}.\label{eq-covr}
\end{multline} A further simplification is possible by defining
\begin{equation}
\rho_{ijkl}:=\frac{\sigma_{ijkl}}{\sqrt{\sigma_{ii}\sigma_{jj}\sigma_{kk}\sigma_{ll}}}.
\label{eq-normcor}
\end{equation} Equation \eqref{eq-covr} becomes \begin{align}
n\text{COV}(\ul{r}_{ij},\ul{r}_{kl})=\rho_{ijkl}&-\frac12\rho_{kl}(\rho_{ijkk}+\rho_{ijll})-\frac12
\rho_{ij}(\rho_{iikl}+\rho_{jjkl})\notag\\&+\frac14\rho_{ij}\rho_{kl}(\rho_{iikk}+\rho_{iill}+\rho_{jjkk}+\rho_{jjll}).\label{eq-ihsh}
\end{align} Equation \eqref{eq-ihsh} has been rediscovered every 33
years by successive generations of statisticians (Isserlis
(\citeproc{ref-isserlis_16}{1916}), Hsu (\citeproc{ref-hsu_49}{1949}),
Steiger and Hakstian (\citeproc{ref-steiger_hakstian_82}{1982})).

\sectionbreak

\section{Discussion}\label{discussion}

Discrete, continuous. Random variables and realizations. Holland
(\citeproc{ref-holland_79}{1979}), Gifi
(\citeproc{ref-gifi_B_90}{1990}), Ouimet
(\citeproc{ref-ouimet_21}{2021})

\sectionbreak

\section*{References}\label{references}
\addcontentsline{toc}{section}{References}

\phantomsection\label{refs}
\begin{CSLReferences}{1}{0}
\bibitem[\citeproctext]{ref-gifi_B_90}
Gifi, A. 1990. \emph{Nonlinear Multivariate Analysis}. New York, N.Y.:
Wiley.

\bibitem[\citeproctext]{ref-hemelrijk_66}
Hemelrijk, J. 1966. {``{Underlining Random Variables}.''}
\emph{Statistica Neerlandica} 20: 1--7.

\bibitem[\citeproctext]{ref-holland_79}
Holland, P. W. 1979. {``{The Tyranny of Continuous Models in a World of
Discrete Data}.''} \emph{IHS-Journal} 3: 29--42.

\bibitem[\citeproctext]{ref-hsu_49}
Hsu, P. L. 1949. {``The Limiting Distribution of Functions of Sample
Means and Application to Testing Hypotheses.''} In \emph{Proceedings of
the First Berkeley Symposium on Mathematical Statistics and
Probability}, edited by Neyman J, 359--401. University of California
Press.

\bibitem[\citeproctext]{ref-isserlis_16}
Isserlis, L. 1916. {``On Certain Probable Errors and Correlation
Coefficients of Multiple Frequency Distributions with Skew
Regression.''} \emph{Biometrika} 11 (3): 185--90.

\bibitem[\citeproctext]{ref-mann_wald_43}
Mann, H. B., and A. Wald. 1943. {``{On Stochastic Limit and Order
Relationships}.''} \emph{Annals of Mathematical Statistics} 14: 217--26.

\bibitem[\citeproctext]{ref-ouimet_21}
Ouimet, F. 2021. {``General Formulas for the Central and Non-Central
Moments of the Multinomial Distribution.''} \emph{Stats} 4: 18--27.

\bibitem[\citeproctext]{ref-steiger_hakstian_82}
Steiger, J. H., and A. R. Hakstian. 1982. {``{The Asymptotic
Distribution of Elements of a Correlation Matrix: Theory and
Application}.''} \emph{British Journal of Mathematical and Statistical
Psychology} 35: 208--15.

\end{CSLReferences}




\end{document}
